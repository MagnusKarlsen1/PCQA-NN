\section{Future Works}
The model showed promising results, however there is some room for improvement. The improvements we think are most logical as the next step are presentedin this section, ranked after importance:

\begin{enumerate}
  \item \textbf{Improve accuracy of labels for imperfect data}
  The strategy for generating updated labels for points near holes has proven to be an effective tool for learning to identify holes. To further refine this approach, we propose further imrpoving this idea, by more accurately estimating the new surface densities when introducing holes and noise into point clouds. This can be achieved by computing the points inside feature for both the original (perfect) and modified (imperfect) point clouds. The ratio between the two can then be used to scale the surface density labels individually for each point in the imperfect point cloud.

  \item \textbf{More diverse training data}
  While there is always room for improving data generation, we specifically identify the need to increase the diversity and geometric complexity of the shapes used, as well as the variety of meshing tools applied. Although testing on the chainwheel geometry—more complex than any shape in the training set—did not reveal major issues, it is reasonable to assume that broader geometric diversity could enhance the model’s ability to learn geometry bias. The Poisson disk sampling method was selected as the primary meshing tool due to its ability to maintain approximately uniform surface density while preserving a degree of randomness, in contrast to deterministic methods like isometric meshing. Nevertheless, it is likely that all meshing tools introduce some form of bias. This is supported by observations in Figure~\ref{fig:sd_mesh}, where identical meshing conditions yielded a slightly higher mean surface density and lower standard deviation for the most homogeneous shape (the sphere), suggesting a tool-induced bias.
  
  This is something that can always be improved, however we see a need for spicifically improving the diversity and complexity of the geometries uesed, and the diversity in meshing tools. It did not seem to be a problem, when testing on the chainwheel geometry, which is more complex than any of the training geometries, but still it would be logial to introduce a higher diversity of different geometries to the dataset to improve the learning of the geometry bias. Poissons disc was used as meshing tool beacause it generates somewhat consistent surface density through the pointcloud while keeping a random element to the pattern in oposition to e.g. isometric meshing tools. However it is believed that any meshing tool might introduce some sort of bias, which can also be seen in Figure~\ref{fig:sd_mesh} where the exact same conditions for meshing tended to result in slightly higher mean surface density and slightly lower standard deviation for the most homogeneous shape (the ball).

  \item \textbf{Investigate different features}
  While the selected features performed well during training, there is still potential omprovements in to be made in feature selection and representation. This could be exploring additional descriptors such as sphericity and anisotropy, as well as enhancing the current features. One way to increase the information captured by the current features is to compute them across multiple neighborhood sizes. This approach provides insight into local geometric variation within the neighborhood. Neighborhoods could also be defined relative to the average radius to maintain scale consistency. Additionally, the gradient difference feature may benefit from alternatives such as including gradients of multiple neighbors or using the standard deviation of gradients within each neighborhood.
\end{enumerate}