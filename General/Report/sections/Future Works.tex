\section{Future Works}
In this section, suggestions for further improving the method are presented in order based on their expected impact.

\begin{enumerate}
  \item \textbf{Improve accuracy of labels for imperfect data}
  The strategy for generating updated labels for points near holes has proven to be an effective tool for learning to identify holes. To further refine this approach, we propose further imrpoving this idea, by more accurately estimating the new surface densities when introducing holes and noise into point clouds. This can be achieved by computing the points inside feature for both the original (perfect) and modified (imperfect) point clouds. The ratio between the two can then be used to scale the surface density labels individually for each point in the imperfect point cloud.

  \item \textbf{More diverse training data}
  While there is always room for improving training data, we specifically suggest increasing the diversity and complexity of geometries to enhance the models ability to learn geometry bias, as well as intorducing a variaty of meshing tools to avoid single meshing tool bias.

  \item \textbf{Investigate different features}
  While the selected features performed well during training, there is still potential improvements in feature selection and representation. This could be exploring additional geometry features such as sphericity and anisotropy, as well as enhancing the current features. One way to increase the information captured by the current geometry features is to compute them across multiple neighborhood sizes. This approach provides insight into local geometric variation within the neighborhood. Neighborhoods could also be defined relative to the average radius to maintain scale consistency with volume density features. Additionally, the gradient difference feature may benefit from exploring alternative representations such as including gradients of multiple neighbors or using the standard deviation of gradients within each neighborhood.
\end{enumerate}