\section{Discussion}
While the model showed promising results, there are several limitations and areas to improve.

\begin{itemize}
    \item \textbf{Limited geometric diversity}
    The training set lacked complex and varied geometries. Although the model performed well on the more complex chainwheel geometry, which exceeded the complexity of training shapes, it is still likely that broader geometric diversity would improve generalization.
    \item \textbf{Limitations in imprefect labels}
    Figure~\ref{fig:pred_vs_label_better} showes a great accuracy in mean and standard deviation across different meshsizes. Especially when considering the variation in surface density within a meshsize from the meshing tool seen in Figure~\ref{fig:sd_mesh}. However, a slight tendency to underestimate surface density can be observed. This bias likely stems from the simplified labeling of points near holes, where a fixed value of half the surface density was used. While effective for learning the concept of missing data, it does not reflect the true variation near holes and may lead to systematic underestimation. In fact, the intitial model was trained only on perfect data and achieved higher accuracy on perfect data, but performed poorly on test data with holes.
    \item \textbf{Single meshing tool bias}
    The Poisson disk sampling method was selected as the primary meshing tool due to its ability to maintain approximately uniform surface density while preserving a degree of randomness, in contrast to methods like isometric meshing. Nonetheless, it is likely that relying on a single tool for meshing may introduce systematic bias. This is supported by observations in Figure~\ref{fig:sd_mesh}, where identical meshing conditions yielded a slightly higher mean surface density and lower standard deviation for the most homogeneous shape (the ball), suggesting a tool-induced bias for ceartain certain geometries. Figure~\ref{fig:holes_predict} also illustrates this bias in the predictions, particularly in the control plot with 0 holes. While the variation is minor, it appears to correlate with the underlying geometry. However, this pattern could also be influenced by other factors than the meshing tool.
    \item \item \textbf{Feature limitations}
    While the selected features performed well during training, their current implementation may not fully capture complex local geometries. In particular, using a single neighborhood size could mask variations between the core and periphery of a neighborhood. On top of that, geometry features are based on a set neighborhood sizes, while volume density features are based on a set radius. This may cause inconsistence between the two categories of features.
\end{itemize}