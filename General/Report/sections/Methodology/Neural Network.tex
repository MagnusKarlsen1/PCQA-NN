<\subsection{Neaural Network}
To assess the quality of a point cloud, we utilized a neural network. The network’s role is to uncover the relationship between the extracted features and the surface density across a variety of shapes and geometries. In considering the most suitable architecture, both XGBoost and a Multi-Layer Perceptron (MLP) were evaluated, as both are well-suited for regression problems.

XGBoost, known for its speed and effectiveness on structured data, initially provided a solid baseline. However, while it performed well on simpler geometries, it struggled to capture the more complex patterns within the features that are critical for accurately estimating surface density. As a result, the predictions from XGBoost, while reasonable, were not sufficient for the precision required in this task.

In contrast, the MLP’s capacity to learn intricate, non-linear relationships enabled it to identify much more subtle and complex correlations within the data. This ability translated into more accurate and reliable predictions, particularly for geometrically complex or irregular areas of the point cloud.

Initial testing confirmed that while XGBoost was faster and reasonably effective, the MLP consistently outperformed it by producing more nuanced predictions and reducing false estimations. Consequently, the MLP was chosen for this work, balancing accuracy and the flexibility needed to handle diverse and challenging point cloud geometries.



