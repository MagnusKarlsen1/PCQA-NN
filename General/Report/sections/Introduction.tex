\section{Introduction}
Point clouds serve as a foundational data structure in robotic 3D scanning, enabling accurate geometric reconstruction of objects and environments. The quality of these point clouds is critical for downstream tasks such as object modeling, inspection, localization, and navigation. A reliable metric for local point cloud quality assessment (PCQA) is therefore indispensable, especially in automated systems that must decide whether a re-scan is warranted in real-time. Existing methods often rely on global point density or volume-based metrics which are computationally efficient but poorly reflect surface-level fidelity.

This report proposes a novel, data-driven approach for estimating local surface density — a more faithful representation of point cloud quality — using neural networks. While surface density has been recognized as superior to volume density for capturing local structure, its direct computation is computationally expensive and infeasible for real-time or large-scale scanning applications. The proposed method overcomes this limitation by learning to infer surface density from a set of rapidly computable features that include both local volume density and geometric bias descriptors.

To ground this concept, a 2D analogy is presented where the limitations of area-based density measures in capturing edge fidelity are illustrated. This motivates the extension of the concept to 3D, where volume density — analogous to area density in 2D — serves as a rough but computable approximation of true surface fidelity. The core idea of this methodology is to leverage this correlation and predict surface density indirectly by using a neural network trained on volume-based features augmented with geometry-aware descriptors.

The ultimate goal is to produce a real-time capable PCQA mechanism that retains high sensitivity to quality-relevant features such as surface curvature, holes, and geometric distortions — enabling smarter, self-optimizing robotic scanning systems.

What distinguishes the method proposed in this report is its specific focus on surface density as the target metric and its hybrid feature approach, combining:

Efficiently computable local volume densities (adaptive-radius based),

Geometry-sensitive descriptors (e.g., curvature, linearity, planarity),

Gradient-based metrics for capturing surface variance.

By identifying and learning the geometry bias that distorts volume-based estimations, this methodology bridges the gap between computational tractability and fidelity, offering a novel pathway toward scalable, real-time PCQA in robotic systems.